\documentclass[conference]{IEEEtran}
\IEEEoverridecommandlockouts
% The preceding line is only needed to identify funding in the first footnote. If that is unneeded, please comment it out.
\usepackage{cite}
\usepackage{amsmath,amssymb,amsfonts}
\usepackage{algorithmic}
\usepackage{graphicx}
\usepackage{textcomp}
\usepackage{xcolor}
\def\BibTeX{{\rm B\kern-.05em{\sc i\kern-.025em b}\kern-.08em
    T\kern-.1667em\lower.7ex\hbox{E}\kern-.125emX}}
\begin{document}

\title{Assignment 4}

\author{\IEEEauthorblockN{1\textsuperscript{st} Raymond 'Akau'ola - raka143}
\IEEEauthorblockA{\textit{Department of Electrical and Computer Engineering} \\
\textit{University of Auckland}\\
raka143@aucklanduni.ac.nz}}

\maketitle

\section{Ranking}
List of designs in decreasing order of rank:
\begin{enumerate}
	\item Design 1001
	\item Design 1036 == Design 1042
	\item Design 1025
	\item Design 1019
	\item Design 1020
\end{enumerate}

\section{Methodology}
In order to rank the designs according to the goodness of their object-oriented design, pairs of designs were compared empirically. Metrics were used as a reflective indicator as to whether the empirical relationship between designs held or not i.e. to confirm the relationship. 

To compare two designs empirically based on the goodness of their object-oriented design, the sets objects created from both are compared. The goodness of an object-oriented design depends on the objects and whether the objects created accurately model real world concepts with clear state, behaviour and identity, as well as making sense within the context of the problem being addressed. Therefore, comparing the sets of objects between two designs using this criteria will effectively provide the empirical relationship between them.

To ease the process of ranking the designs, the amount of pairs needed to be compared were reduced by dividing the designs into two sub-groups using the average CBO (Coupling between Objects) value, with the designs with similar average CBO values being grouped together. This metric measures the number of classes to which a class is coupled with and is indicative of how reusable and maintainable a class is. This can also be a reflective indicator of good object-oriented design because objects that are reusable in other contexts can indicate that the real world concept it represents is accurate. 

\section{Justification}
With the given designs, the two sub-groups were formed:
\begin{enumerate}
	\item Design 1001, 1036, 1042
	\item Design 1019, 1020, 1025
\end{enumerate}
This means that the designs within sub-group 1 are empirically better than those in sub-group 2. 

\subsection{Comparing Design 1001 and Design 1036}
Both designs incorporate objects which encompass the concept of a board, however, in Design 1036, objects are created to represent pits and mancala pits which are both relevant to the context schema but could have both been included within MancalaBoard object. The context schema does not necessarily require all these objects to be handled separately. When Pit and MancalaPit objects are used, it is used with MancalaBoard so it would make more sense for both these concepts to be encompassed in a single object. In Design 1001, there is a similar occurrence with TurnManager, which is an object that carries state of the current players turn, which could be placed in BoardManager or Board object, however, the TurnManager object can be required to be dealt with independently of the board. Therefore, Design 1001 is a better object-oriented design than Design 1036.

\subsection{Comparing Design 1036 and Design 1042}
Both designs create objects that correspond to the concept of a Board and a Pit/House, however, Design 1036 contains a MancalaPit object which carries the respective owners score. This is referred to by the context schema but the concept of the store and a pit are similar enough that it could be encompassed using only a Pit object. Similarly, the House object in Design 1042 is related to the KalahBoard object however these objects are not required to be dealt with separately so those responsibilities and state in House object can be included within KalahBoard. Both designs are similar in the objects that they create and the responsibilities of each object, and they both make sense within the context schema. Therefore, Design 1036 and 1042 have similar rankings.

\subsection{Comparing Design 1019 and Design 1020}



%\begin{thebibliography}{00}
%\bibitem{b1} G. Eason, B. Noble, and I. N. Sneddon, ``On certain integrals of Lipschitz-Hankel type involving products of Bessel functions,'' Phil. Trans. Roy. Soc. London, vol. A247, pp. 529--551, April 1955.
%\bibitem{b2} J. Clerk Maxwell, A Treatise on Electricity and Magnetism, 3rd ed., vol. 2. Oxford: Clarendon, 1892, pp.68--73.
%\bibitem{b3} I. S. Jacobs and C. P. Bean, ``Fine particles, thin films and exchange anisotropy,'' in Magnetism, vol. III, G. T. Rado and H. Suhl, Eds. New York: Academic, 1963, pp. 271--350.
%\bibitem{b4} K. Elissa, ``Title of paper if known,'' unpublished.
%\bibitem{b5} R. Nicole, ``Title of paper with only first word capitalized,'' J. Name Stand. Abbrev., in press.
%\bibitem{b6} Y. Yorozu, M. Hirano, K. Oka, and Y. Tagawa, ``Electron spectroscopy studies on magneto-optical media and plastic substrate interface,'' IEEE Transl. J. Magn. Japan, vol. 2, pp. 740--741, August 1987 [Digests 9th Annual Conf. Magnetics Japan, p. 301, 1982].
%\bibitem{b7} M. Young, The Technical Writer's Handbook. Mill Valley, CA: University Science, 1989.
%\end{thebibliography}

\end{document}
